\documentclass[11pt]{beamer}
\usepackage{graphics}
\usepackage{color}
\usetheme{Pittsburgh}
%\usetheme{Antibes}
%\setbeamercovered{transparent}
\usefonttheme{serif}
\setbeamercolor{itemize item}{fg=brown}
\setbeamercolor{normal text}{fg=black}

\definecolor{pp}{RGB}{117,107,177}
\definecolor{bb}{RGB}{50,145,195}

\usepackage{amsthm}
\setbeamertemplate{theorems}[numbered]
\theoremstyle{definition}
\newtheorem{defi}{Definition}


%\usecolortheme{beaver}
% Options: albatross crane beetle dove fly seagull wolverine beaver
% \usepackage{helvet}
% options serif avant bookman chancery charter euler helvet mathtime mathptm mathptmx newcent palatino pifont utopia

% \item<1-> Means from slide one on
% \item<1-3> Means display item from slide one to three
% \alert<2> Alert on slide 2

%\begin{block}{Markets don't exists in a void} \end{block}

\title[Economics 970]{Economics 970: Sophomore Tutorial}
\subtitle[]{Review}
\author[J Zilinsky]{Jan Zilinsky}
\date{Harvard University}
\institute[Harvard]{Experiments in Economics}


\begin{document}

\frame{
\maketitle
}

\frame{\frametitle{Topics}
\begin{enumerate}
\item Social pressure
\item Donations and charitable/altruistic behavior 
\item Pure altruism vs. warm glow
\item Peer pressure and social comparisons
\item Deception
\item Reputation
\item Coordination problems
\item Competitiveness and confidence
\item Gender
\item Use of credit cards
\item Development
\end{enumerate}
}

\frame{\frametitle{Themes}

\begin{itemize}
\item Effects of framing on decision-making
\item Information overload, confusion
\item Anonymity vs. scrutiny
\item Field environments versus experiments in the lab
\end{itemize}
}




\frame{\frametitle{Experimental economics vs. social psychology}

\begin{itemize}
\item Economists
\begin{enumerate}
\item Study markets
\item Downplay the importance of context
\item Often assume all people have the same preferences
\item Are interested in equilibria
\item Ask what happens to an environment is prices (or other constraints) change
\item Are inclined to view outcomes (if freely chosen) are ultimately Pareto efficient
\end{enumerate}
\item Psychologists study people (character traits, belief and identity formation, etc.) and see their subjects as prone to making mistakes
\item Experimental economists are in some sense in between
\end{itemize}
}

\frame{\frametitle{Features of field experiments}

\begin{itemize}
\item Natural environment
\item A more representative subject pool
\item Realistic / properly-sized stakes
\item Randomization
\item (Ecological validity under some assumptions; do subjects know they are observed?)
\end{itemize}

The first experiment we read about covered social pressure and contributions to low-income students.
}



\frame{\frametitle{Happiness}

\begin{itemize}
\item What is it? (E.g. achieving a rational life-plan, pursuit of valuable/enjoyable activities)
\item What it is not: comparisons with fortunate neighbors
\item A paternalistic view: ``People do not necessarily know what will give them lasting satisfaction''
\item Predictors of happiness
\begin{enumerate}
\item Attitudes: optimism
\item Income/wealth
\item Employment
\item Relationships with family and friends
\item Status
\end{enumerate}
\end{itemize}
}

\frame{\frametitle{Where do preferences come from?}

\begin{itemize}
\item Indoctrination: the process by which organizations imbue society with their ideology or opinion
\item Peer effects: influence of the decisions of other’s on our own choices
\end{itemize}
}

\frame{\frametitle{Peer effects}

\begin{itemize}
\item Productivity of workers is influenced by the presence of peers 
\item Performance tends to converge: having high-productivity friends or co-workers is valuable
\item We seem to be energized by the presence of others (if they are motivated)
\item Less motivated individuals are pressured to work harder
\end{itemize}
}

\frame{\frametitle{Collective action and social norms (Akerlof)}

\begin{itemize}
\item Our training/ideology shapes how we interpret games
\item Coase: efficient allocation will arise if we assign property rights
\item Akerlof: public goods will not be overused if people feel they have a sense of duty to behave in certain ways
\item Social norms survive if those who break the rules are punished 
\end{itemize}
}

\frame{\frametitle{Gender}

\begin{itemize}
\item The topic fundamentally involves questions of identity 
\item Boys run faster if they run with other boys
\item Existing research still emphasizes markets
\item Maze-solving: under a  winner-takes-all scheme, men try harder
\item But competitiveness is not universally higher among men
\item Women share more than men even if generosity is costly (the demand curves for altruism cross)
%\item The majority of women appears sends honest messages in a deception game
\item But remember: publication bias (and implications)
\end{itemize}
}


\frame{\frametitle{Ongoing work}

\begin{itemize}
\item Identity priming 
\item Persuasion without information provision
\item Non-Bayesian updating
\item Stereotypes
\end{itemize}
}



\end{document}